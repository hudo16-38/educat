\chapter{Geometria}
\label{ch:geom}

\begin{example}{}{t9-25-04}
	V trojuholníkovej sieti na obrázku vyfarbujeme trojuholníky podľa nasledujúcich pravidiel:
	V 1. kroku vyfarbíme jeden trojuholník. \\
	V 2. kroku vyfarbíme len tie trojuholníky, ktoré majú spoločnú stranu s trojuholníkom
	vyfarbeným v prvom kroku. \\
	V každom ďalšom kroku vyfarbíme všetky trojuholníky, ktoré majú aspoň jednu spoločnú stranu
	s trojuholníkmi vyfarbenými v predchádzajúcom kroku. \\
	\includegraphics[width=\linewidth]{obrazky/t9-25-04.png}
	Koľko trojuholníkov bude vyfarbených po piatich krokoch?
\end{example}


\begin{example}{}{t9-25-mapa}
	Adam si nakreslil mapu svojich najčastejších vychádzok a do tabuľky si poznačil vzdialenosti
	jednotlivých miest v krokoch. \\
	\includegraphics{obrazky/t9-25-mapa1.png}\\
	\space \\
	\includegraphics{obrazky/t9-25-mapa2.png} \\
	
	1. Na základe údajov v obrázku urči veľkosť ostrého uhla medzi cestou z kostola do obchodu
	a cestou z kostola do domu. Veľkosť uhla zapíš v stupňoch. \\
	2. Na základe údajov o počte krokov v tabuľke vypočítaj, koľko krokov Adam urobí, ak pôjde
	od kostola do obchodu najkratšou cestou.
\end{example}

\begin{example}{}{t9-25-lieky}
	Plastový obal s hliníkovou fóliou, ktorý slúži na balenie liekov, sa nazýva blister. Blistre sú balené
	do papierových škatuliek. Pôvodne bolo v jednom blistri 10 tabletiek. V jednej papierovej škatuľke
	sa predávalo 90 kusov tabletiek (obr. 1). Výrobca sa rozhodol šetriť životné prostredie. Iným
	rozmiestnením tabletiek sa mu podarilo zabaliť do jedného blistra až 15 kusov tabletiek (obr. 2). \\
	\includegraphics[width=\linewidth]{obrazky/t9-25-lieky.png} \\
	1. O koľko menej blistrov bude v jednej škatuľke pri novom rozmiestnení tabletiek, ak počet
	tabletiek v jednej škatuľke zostane nezmenený? \\
	2. O koľko cm sa zmenší povrch novej škatuľky, ak sa jej šírka zmení z pôvodných 5,7 cm
	na 3,8 cm? 
\end{example}

\begin{example}{}{t9-25-tatiana}
	Tatiana rada behá v parku okolo piatich rovnakých kruhových záhonov s kvetmi. Na obrázku je
	zvýraznená jej každodenná trasa. Koľko metrov meria jej trať? Počítaj s hodnotou $\pi = 3,14$.
	Šírku bežeckej cestičky zanedbávame. \\
	\includegraphics[width=\linewidth]{obrazky/t9-25-tatiana.png}
	
\end{example}

\begin{example}{}{t9-25-podorys}
	Na obrázku je znázornený pôdorys bytu. V tabuľke sú uvedené rozmery niektorých miestností
	v metroch. Betónový strop sa nachádza vo výške 260 cm. Steny aj okná majú tvar obdĺžnika. Byt má
	tri okná orientované na juh. \\
	\includegraphics[width=\linewidth]{obrazky/t9-25-podorys.png} \\
	1. Majiteľ bytu chce v obývacej izbe spojenej s kuchyňou vymaľovať jednu stenu, na ktorej je okno.
	Rozmery okna sú 2 m a 1,5 m. Koľko metrov štvorcových je potrebné vymaľovať? Výsledok
	uveď s presnosťou na desatiny. \\
	2. Koľko metrov kubických vzduchu sa zmestí do prázdnej spálne, ak je jej strop pre osvetlenie
	znížený o 20 cm? Výsledok uveď s presnosťou na desatiny. 
\end{example}

\begin{example}{}{t9-25-stavebnica}
	V stavebnici sú drevené paličky rôznej dĺžky. Dĺžky paličiek sú v celých centimetroch, pričom
	najkratšie majú dĺžku 2 cm. Z každého druhu paličiek sa dá poskladať štvorec so stranou dĺžky 18 cm
	tak, že použijeme len paličky rovnakej dĺžky. \\
	1. Koľko paličiek dlhých 3 cm je potrebných na poskladanie štvorca so stranou dĺžky 18 cm? \\
	2. Koľko rôznych dĺžok môžu mať paličky v stavebnici?
\end{example}

\begin{example}{}{t9-25-25}
	Stavba na obrázku sa skladá z troch rovnakých kociek s dĺžkou hrany 4 cm. Ema chce postaviť
	ďalšiu takú istú stavbu s rovnakými rozmermi z menších kociek. Použila dve kocky
	s dĺžkou hrany 3 cm a osem kociek s dĺžkou hrany 2 cm. Koľko kociek s dĺžkou hrany 1 cm bude
	potrebovať na dokončenie takejto stavby, ak novú kocku priloží vždy celou stenou k stene
	kocky, ktorá už v stavbe je? \\
	\includegraphics[width=\linewidth]{obrazky/t9-25-25.png}
\end{example}

\begin{example}{}{t9-25-26}
	Na obrázku sú znázornené tri kružnice: $k_1(S_1 , 4 cm)$, $k_2(S_2 , 5 cm)$, $k_3(S_3 , 3 cm)$. Jednotlivé dvojice kružníc sa vzájomne dotýkajú v bodoch $A$, $B$ a $C$. \\
	\includegraphics[width=\linewidth]{obrazky/t9-25-26.png} \\
	Vypočítaj obvod trojuholníka $S_1S_2S_3$. 
\end{example}

\begin{example}{}{t9-25-27}
	Na obrázku je znázornená priamka AB a bod M. Dokresli do obrázka rovnoramenný
	lichobežník ABCD so základňou AB tak, aby bod M ležal v strede ramena AD. \\
	\includegraphics[width=\linewidth]{obrazky/t9-25-27.png} \\
	Ktoré z tvrdení o lichobežníku ABCD je nepravdivé? 
	\begin{enumerate}
		\item Strany lichobežníka AB a CD sú rovnobežné.
		\item Bod M je rovnako vzdialený od bodu B aj od bodu C.
		\item Výška lichobežníka je menšia ako dĺžka strany BC.
		\item Uhly BAD a ABC sú zhodné.
	\end{enumerate}
\end{example}

\begin{example}{}{t9-25-30}
	Teplomer na obrázku spoľahlivo meria teplotu vzduchu na vnútornej stupnici v stupňoch Celzia (°C) a na vonkajšej stupnici v stupňoch Fahrenheita (°F). Ktoré z nasledujúcich tvrdení je nepravdivé?
	\begin{enumerate}
		\item Teplomer ukazuje teplotu približne 15 °C.
		\item Teplomer ukazuje teplotu približne 60 °F.
		\item Ak teplota klesne o 40 °F, bude teplomer ukazovať približne 20 °C.
		\item Ak teplota klesne o 40 °C, bude teplomer ukazovať približne –13 °F.
	\end{enumerate}
	\includegraphics[width=\linewidth]{obrazky/t9-25-30.png}
	
\end{example}