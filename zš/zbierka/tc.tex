\chapter{Teória čísel}
\label{ch:tc}

\begin{example}{}{t9-25-17}
	Dopravné lietadlá lietajú vo výškach okolo 10 000 m. V týchto výškach je riedky vzduch a sú tam
	veľmi nízke teploty. V kabíne lietadla namerali teplotu vzduchu 21,7 °C, mimo lietadla –55,4 °C.
	Rozdiel teploty v kabíne lietadla a teploty mimo lietadla je: 
\end{example}

\begin{example}{}{t9-25-20}
	Mama upiekla koláč a nechala ho na plechu vychladnúť. Najprv prišla Eva a zjedla z neho jednu
	štvrtinu. Neskôr prišiel Daniel a zjedol jednu štvrtinu zo zvyšku koláča. Aká časť koláča zostala
	na plechu?
\end{example}