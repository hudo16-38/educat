\chapter{Pravdepodobnosť a štatistika}
\label{ch:pas}

\begin{example}{}{t9-25-05}
	Starý otec má 10 vnúčat. Každému kúpil gumu na gumovanie v tvare sovy. Štyri z nich boli žlté,
	jedna biela, tri modré a dve zelené. Deti si svoju gumu postupne vyberali z nepriehľadného
	vrecúška a čo si vytiahli, to už do vrecúška nevrátili. Janko si vybral gumu ako tretí v poradí.
	Aká je pravdepodobnosť, že si vybral modrú sovu, ak dve sesternice pred ním si vybrali zelenú
	a modrú? Výsledok uveď v percentách. 
\end{example}

\begin{example}{}{t9-25-smartwatch}
	Martin a jeho sestra Laura dostali na narodeniny inteligentné hodinky. Každý deň na nich sledujú
	počet prejdených krokov. V diagrame sú zaznamenané počty krokov, ktoré Martin urobil počas
	jedného týždňa. \\
	\includegraphics[width=\linewidth]{obrazky/t9-25-smartwatch.png}\\
	1. Vypočítaj priemerný denný počet krokov Martina v tomto týždni.\\
	2. Aplikácia v hodinkách dáva užívateľovi aj rôzne užitočné rady. Jednou z nich je: „Prejsť denne
	8 000 krokov je prospešné pre zdravie, ale prejsť aspoň 110 krokov za 1 minútu je oveľa lepšie.“
	Podľa záznamov z Martinových a Lauriných hodiniek posúď, kto z nich prešiel priemerne aspoň
	110 krokov za minútu. \\
	\includegraphics[width=\linewidth]{obrazky/t9-25-smartwatch-zaznam.png}
	
\end{example}

\begin{example}{}{t9-25-24}
	
	Alena sa opýtala 80 detí na ich obľúbené zvieratko. Údaje v percentách, ktoré zistila, znázornila
	v kruhovom diagrame. Jeden údaj zo svojho prieskumu zabudla doplniť. Koľko detí uviedlo, že ich najobľúbenejším zvieratkom je korytnačka? \\
	\includegraphics[width=\linewidth]{obrazky/t9-25-24.png}
\end{example}