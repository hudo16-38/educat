\chapter{Lineárne rovnice, nerovnice a výrazy}
\label{ch:rovnice}

\begin{example}{}{}
	Vypočítaj hodnotu výrazu $3x + 2y + z$, ak $x = 5$, $y = 2$ a $z = 8$.
\end{example}

\begin{example}{}{}
	Vlak idúci priemernou rýchlosťou 80 km/h prejde svoju trasu za 3,5 hodiny. Po rekonštrukcii
	trate sa priemerná rýchlosť vlaku zvýši na 100 km/h. Koľko hodín bude trvať cesta vlakom
	po zrekonštruovanej trati na tej istej trase? Výsledok uveď s presnosťou na desatiny. 
\end{example}

\begin{example}{}{t9-25-08}
	Ak do automatu vložíme číslo x = 870, ktoré číslo z neho podľa schémy vypadne? \\
	\includegraphics[width=\linewidth]{obrazky/t9-25-08.png}
\end{example}

\begin{example}{}{t9-25-09}
	Model auta je oproti skutočnosti zmenšený v pomere 1 : 50. Vypočítaj skutočnú dĺžku tohto auta,
	ak dĺžka modelu je 8 cm. Výsledok uveď v metroch.
\end{example}

\begin{example}{}{t9-25-18}
	Daný je výraz $T = 20a – (8a : 4) \cdot 2 – 5$. V ktorej možnosti sa výraz $T$ rovná výrazu $U$?
	\begin{enumerate}
		\item $U = 6a – 5$
		\item $U = 16a – 5$
		\item $U = 11a$
		\item $U = 36a – 5$
	\end{enumerate} 
\end{example}

\begin{example}{}{t9-25-20}
	Juraj sa rozhodol, že si bude každý mesiac odkladať na elektrický bicykel. Rodičia mu sľúbili,
	že k nasporenej sume mu pridajú 10 \%. Juraj si po celý rok mesačne odkladal 200 € a rodičia
	mu prispeli tak, ako sľúbili. V obchode ponúkajú 4 rôzne bicykle. V ktorej možnosti je cena
	najdrahšieho bicykla, ktorý si Juraj môže kúpiť?
	\begin{enumerate}
		\item 2 200 €
		\item 2 400 €
		\item 2 600 €
		\item 2 900 €
	\end{enumerate}
\end{example}

\begin{example}{}{t9-25-vaha}
	Na misky rovnoramenných váh sme položili niekoľko guliek dvoch druhov tak, že misky
	sú v rovnováhe. Guľky jedného druhu majú rovnakú hmotnosť. \\
	\includegraphics[width=\linewidth]{obrazky/t9-25-vaha.png}\\
	Koľko čiernych guliek má rovnakú hmotnosť ako šesť bielych?
\end{example}

\begin{example}{}{t9-25-28}
	V ktorej možnosti je riešenie nasledujúcej rovnice?
	\begin{equation*}
		\frac{6x-5}{4} = 2x - 3
	\end{equation*}
	\begin{enumerate}
		\item -8,5
		\item 2
		\item 3,5
		\item 4
	\end{enumerate}
\end{example}

\begin{example}{}{t9-25-29}
	Cyklista prešiel v júni 500 kilometrov. Prejdenú vzdialenosť plánuje zvyšovať o 20 \% mesačne.
	Koľko kilometrov by mal prejsť v septembri toho istého roka?
\end{example}