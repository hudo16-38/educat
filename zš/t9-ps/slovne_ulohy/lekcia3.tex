\documentclass{article}

\usepackage[utf8]{inputenc}
\usepackage{amsthm}
\usepackage[slovak]{babel}
\newtheorem{example}{Príklad}

\title{Lekcia 3 - Priama a nepriama úmernosť a percentá}
\date{\today}
\author{Educat - vzdelávacie centrum}

\begin{document}
	\maketitle
	
	\begin{example}
		4 stavbári postavia múr za 10 hodín. Za koľko ho postavia 10 stavbári?
	\end{example}
	
	\begin{example}
		5 stavbári z firmy A postavia dom za 12 dní. 7 stavbári z firmy B ho postavia za 10 dní. V ktorej firme majú výkonnejších stavbárov? 
	\end{example}
	
	\begin{example}
		Počítač stál pôvodne 400€. Koľko stojí po 20\%-nom zdražení?
	\end{example}
	
	\begin{example}
		5 stavbári postavili 10 stĺpov za pól dňa. Potom sa k ním pridali ďalší 5-ti. Koľko stĺpov za ten deň postavili?
	\end{example}
	
	\begin{example}
		Topánky stáli po zdražení o 30\% 330€. Koľko stáli pôvodne?
	\end{example}
	
	\begin{example}
		Mikina pôvodne stála 60€. Po zľave stála 45€. Koľko \%-ná bola zľava?
	\end{example}
	
	\begin{example}
		Stroj za 2 hodiny vyrobí 12 súčiastok. Koľko ich vyrobí za 10 hodín?
	\end{example}
	
	\begin{example}
		Auto stálo po zdražení 18 000€. Pred zdražením stálo 15 000€. Koľko \%-né zdraženie nastalo u auta?
	\end{example}
	
	\begin{example}
		8 stavbári postavia dom za 16 dní. Za koľko dní ho postavia 24 stabárov?
	\end{example}
	
	\begin{example}
		Počítač stál pôvodne 400€. Potom zdražel o 25\%. Následne zľacnel o 30\%. Koľko stojí teraz?
	\end{example}
	
	\begin{example}
		Smartfón stále pôvodne 900€. Najprv zľacnel o 10\% a neskôr zdražel o 15\%. Koľko stojí teraz.
	\end{example}
	
	\begin{example}
		Auto istú vzdialenosť pri rýchlosti 100 km/h za 1,2 hodiny. O koľko minút menej by to trvalo autu, ak by šlo o 20 km/h rýchlejšie? 
	\end{example}
	
	\begin{example}
		Jožko zje celý obed za 40 minút. Akú časť obeda má zjedenú za 165 sekúnd ?
	\end{example}
	
	\begin{example}
		15 maliarov natrie plot za 280 minút. Za koľko minút natrie polovicu tohto plota 12 maliarov?
	\end{example}
	
	\begin{example}
		Ak sa denne spotrebuje 1,6 ton uhlia, postačí zásoba v kotolni na 42 dní. Na ako dlho postačí zásoba, keď sa denne spáli iba 1,2 ton uhlia ?
	\end{example}
	
	
\end{document}