\documentclass{article}

\usepackage[utf8]{inputenc}
\usepackage{amsthm}
\usepackage[slovak]{babel}
\newtheorem{example}{Príklad}

\title{Lekcia 1 - Teória čísel}
\date{\today}
\author{Educat - vzdelávacie centrum}

\begin{document}
	\maketitle
	
	\begin{example}
		Modrá a červená loď vyrážajú z prístavu o 11:00 hod. Modrá loď vyráža každých 20 minút a červená každých 30 minút. O koľkej sa prvýkrá stretnú?
	\end{example}
	
	\begin{example}
		Jankova babka chce napiecť koláče. Má k dispozícii 24 marhúľ a 32 jabĺk. Koľko najviac koláčov vie Jankova babka napiecť, aby suroviny použila do koláčov rovnomerne (všade použije rovnako veľa)? 
	\end{example}
	
	\begin{example}
		Máme k dispozícii dlaždice s rozmermi 21 a 28 centimetrov. Chceme nimi pokryť plochu v tvare štvorca. Aká najmenšia možná môže byť strana tejto plochy?
	\end{example}
	
	\begin{example}
		Ak učiteľka na telesnej zoradí žiakov do štvorradu a šesťradu, nikto nevystáva. Koľko najmenej môže byť v triede žiakov?
	\end{example}
	
	\begin{example}
		Biela, modrá a červená loď vyrážajú z prístavu o 10:00. Biela vyráža každých 12, modrá každých 15 a červená každých 25 minút. Koľkokrát sa stretnú do 19:00? 
	\end{example}
	
	\begin{example}
		Teta Anička chce napiecť koláče a má 32 sliviek, 44 broskýň a 64 jabĺk. Koľko koláčov vie najviac napiecť? Koľko bude z každej suroviny v 1 koláči? 
	\end{example}
	
	\begin{example}
		Ak učiteľ na telesnej zoradí deti do päťradu a šesťradu, vždy 1 žiak vystane, Koľko najmenej môže byť v triede žiakov?
	\end{example}
	
	\begin{example}
		Máme kvádre s rozmermi 10, 12 a 14 cm. Do akej najmenšej kocky ich vieme naskladať?
	\end{example}
\end{document}