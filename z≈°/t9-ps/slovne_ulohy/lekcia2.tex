\documentclass{article}

\usepackage[utf8]{inputenc}
\usepackage{amsthm}
\usepackage[slovak]{babel}
\newtheorem{example}{Príklad}

\title{Lekcia 2 - Rovnice, pomer a percentá}
\date{\today}
\author{Educat - vzdelávacie centrum}

\begin{document}
	\maketitle
	
	\begin{example}
		Počas zberu papiera triedy na druhom stupni nazbierali dokopy 620 kilogramov papiera. Piataci nazbierali o 20 kg viac ako šiestaci. Siedmaci nazbierali dvakrát viac ako šiestaci. Ôsmaci nazbierali o 10 kg menej ako šiestaci. Deviataci nazbierali o 10 kg viac ako šiestaci.
		Koľko nazbierali jednotlivé triedy?
	\end{example}
	
	\begin{example}
		Na prvom stupni nazbierali dokopy žiaci 140 kilogramov papiera. Pomer, v ktorom ho zozbierali prváci, druháci, tretiaci a štvrtáci bol 4:5:2:3. Koľko nazbierali jednotlivé triedy?
	\end{example}
	
	\begin{example}
		Feriho sestra je dvakrát taká stará ako on. Dokopy majú 30 rokov. Koľko rokov majú?
	\end{example}
	
	\begin{example}
		Hanka bola pred šiestimi rokmi dvakrát taká stará ako jej sestra Janka. Ak teraz majú dokopy 3O rokov, aké sú staré?
	\end{example}
	
	\begin{example}
		Kamaráti Jožo a Adam majú dokopy 33 rokov. Ak je Adam o 3 roky starší od Joža, koľko majú rokov?
	\end{example}
	
	\begin{example}
		Po zdražení o 30\% stojí počítač 390€. Koľko stál pred zdražením?
	\end{example}
	
	\begin{example}
		V obchode A a B predávajú isté botasky. V obchode A stoja po zdražení o 15\% 23€. V obchide B stoja po zdražení o 10\% 25€. Kde pôvodne boli lacnejšie a o koľko?
	\end{example}
	
	\begin{example}
		Zimná bunda stojí po 40\%-nej zľave 36€. Koľko stála pôvodne?
	\end{example}
	
	\begin{example}
		Koľko \% tvorí obsah vpísaného kruhu vo štvorci z tohoto štvorca?
	\end{example}
	
	\begin{example}
		Koľko \% tvorí obsah štvorca z obsahu kruhu jemu opísanému?
	\end{example}
\end{document}.